\documentclass[serif, hyperref={unicode, breaklinks}, xcolor={x11names, psnames,
  dvipsnames, table}, usepdftitle=false]{beamer}

%
% Beamer settings
%

\usetheme{Warsaw}
\usefonttheme{professionalfonts}
\setbeamercovered{dynamic}
\setbeamertemplate{caption}[numbered]

%
% Packages
%

\usepackage[T1]{fontenc}
\usepackage[utf8]{inputenc}
\usepackage[english]{babel}
\usepackage[babel, activate={true, nocompatibility}, final, tracking=true,
kerning=true, spacing=true]{microtype}
\usepackage{cleveref}
\usepackage{subcaption, graphicx}
\usepackage{booktabs}
\usepackage{mathtools, amssymb, xfrac, mathrsfs, cancel, upgreek, bm}
\usepackage{array}
\usepackage[binary-units, detect-all, detect-display-math]{siunitx}
\usepackage{algorithm,varwidth}
\usepackage[noend]{algpseudocode}
\usepackage[operators, primitives]{cryptocode}
\usepackage[style=alphabetic, backend=biber]{biblatex}

%
% Package settings
%

\addbibresource{Bibliography.bib}
\renewcommand*{\bibfont}{\footnotesize}

\hypersetup{
  pdfinfo={
    Title={The MinRank Problem: Survey, Implementation, and One Application},
    Subject={Cryptography},
    Author={Dario Gjorgjevski},
    Keywords={Multivariate Cryptography, MQ Cryptosystems, MinRank, Gröbner
      Bases, Linear Algebra, Algebraic Cryptanalysis}
  }
}

\AtEveryBibitem{%
  \clearfield{url}%
  \clearfield{urldate}%
  \clearfield{doi}%
  \clearfield{edition}%
  \clearfield{pages}%
  \clearfield{volume}%
  \clearfield{editor}%
  \clearfield{issn}
}

\SetTracking{encoding={*}, shape=sc}{40}
\graphicspath{{figure/}}

\DeclareMathOperator{\rank}{rank}

\newcommand{\MR}{\ensuremath{\mathsf{MR}}}
\newcommand{\NP}{\ensuremath{\mathsf{NP}}}

\let\chyperref\cref
\renewcommand{\cref}[1]{\hyperlink{#1}{\chyperref{#1}}}
\let\Chyperref\Cref
\renewcommand{\Cref}[1]{\hyperlink{#1}{\Chyperref{#1}}}

\renewcommand*{\vec}[1]{\bm{\mathrm{#1}}}
\newcommand{\matr}[1]{\bm{\mathrm{#1}}}
\newcommand{\MQ}{\ensuremath{\mathcal{MQ}}}
\newcommand{\STX}{\ensuremath{\overline{\matr{STX}}}}

%
% Document
%

\begin{document}
\title[MinRank]{The MinRank Problem}
\subtitle{Survey, Implementation, and One Application}
\author[D.~Gjorgjevski]{Dario Gjorgjevski\inst{1}\\
  \texttt{\href{mailto:gjorgjevski.dario@students.finki.ukim.mk}
    {gjorgjevski.dario@students.finki.ukim.mk}}}
\institute[FCSE]{\inst{1}Faculty of Computer Science and Engineering\\
  Ss.\ Cyril and Methodius University in Skopje}
\date[2016-01-12]{January 12, 2016}
\logo{\includegraphics[height=0.66cm]{Logo.png}}

\AtBeginSection[]{%
  \begin{frame}
  \frametitle{Outline}
  \tableofcontents[currentsection]
  \end{frame}
}

\begin{frame}
  \titlepage
\end{frame}

\section{Definition and Fundamental Insights}
\subsection{Definition}
\begin{frame}
  \frametitle{Definition of the MinRank Problem}
  The MinRank problem (MR) is a fundamental problem in linear algebra
  of finding a low-rank linear combination of matrices.
  \begin{definition}[MinRank over a field]\label{def:mr}
    Let $\matr{M}_0;\, \matr{M}_1, \ldots, \matr{M}_m$ be matrices in
    $\mathcal{M}_{\eta \times n}(\mathbb{K})$.  The MinRank problem instance
    $\MR\left(m, \eta, n, r, \mathbb{K};\, \matr{M}_0;\, \matr{M}_1, \ldots,
      \matr{M}_m\right)$ asks us to find an $m$-tuple
    $\vec{\upalpha} = (\alpha_1, \ldots, \alpha_m) \in \mathbb{K}^m$ such that
    $$
    \rank\left(\sum\limits_{i = 1}^{m} \alpha_i \matr{M}_i - \matr{M}_0\right) \le
    r\text{.}
    $$
  \end{definition}
  In practice, we have $\mathbb{K} = \mathbb{F}_q$.
\end{frame}

\subsection{Computational Complexity}
\begin{frame}
  \frametitle{Complexity of the MinRank Problem}
  \begin{theorem}[\autocite{BFS99,Cou01}]\label{thm:mr-comp}
    The MinRank problem is \NP-complete.
  \end{theorem}
  \begin{itemize}
  \item MinRank's \NP-completeness is what allows us to use it as an underlying
    problem in a zero-knowledge authentication scheme.
  \item We will also see a connection between MinRank and multivariate quadratic
    (\MQ) cryptosystems.  Interestingly, any system of multivariate polynomial
    equations can be effectively encoded as a \MR{} instance.
  \end{itemize}
\end{frame}

\section{Known Attacks}
\subsection{The Kernel Attack}
\begin{frame}
  \frametitle{Key Idea Behind the Kernel Attack}
  \begin{itemize}
  \item Proposed by~\textcite{GC00}.
  \item Rather than guess a solution, guess its kernel.  If the kernel is
    guessed correctly, the solution can be solved for.
  \item Let
    $H_{\vec{\upbeta}} = \sum\nolimits_{i = 1}^{m} \beta_i \matr{M}_i -
    \matr{M}_0$ ($\vec\upbeta$ is a parameter).
  \item If $\vec\upalpha$ is a solution,
    $\left(\rank H_{\vec\upalpha} \le r \Longleftrightarrow \dim\left(\ker
        H_{\vec\upalpha}\right) \ge n - r\right)$ $\Longrightarrow$ the kernel's
    dimension can be relatively large making guessing more feasible.
  \item Given a correct guess, the solution $\vec\upalpha$ can be retrieved in
    roughly cubic time by simply solving a linear system of equations.
  \end{itemize}
\end{frame}
\begin{frame}
  \frametitle{The Kernel Attack Algorithm}
  \begin{algorithm}[H]
    \renewcommand{\algorithmicrequire}{\textbf{Input:}}
    \renewcommand{\algorithmicensure}{\textbf{Output:}}
    \caption{The Kernel Attack on MinRank}\label{alg:ka}
    \begin{algorithmic}
      \Require $\MR\left(m, \eta, n, r, \mathbb{F}_q;\, \matr{M}_0;\,
        \matr{M}_1, \ldots, \matr{M}_m\right)$
      \Ensure A solution to the \MR{} instance (if any)
      \Repeat
      \State $\vec{x}^{(i)} \sample \mathbb{F}_q^n,\; 1 \le i \le \ceil{\frac m
        \eta}$
      \State $\vec{\upbeta} \gets \text{solve
      }\left\{\left(\sum\nolimits_{j=1}^{m}\beta_j \matr{M}_j -
          \matr{M}_0\right)\vec{x}^{(i)} = \vec{0} \right\},\; 1 \le i \le
      \ceil{\frac m \eta}$
      \Until \begin{varwidth}[t]{\linewidth}
        ($\vec{\upbeta}$ solves the \MR{} instance) $\vee$ (the algorithm has been
        \par run sufficiently many times)
      \end{varwidth}
    \end{algorithmic}
  \end{algorithm}
  Guess \& solve $q^{\ceil{\frac m \eta} r}$ times $\Longrightarrow$
  $\mathcal{O}\left(m \left(\ceil{\frac m \eta} \eta\right)^2 q^{\ceil{\frac m
        \eta} r}\right)$.
\end{frame}
\subsection{Modeling MinRank Instances as \MQ{} Systems}
\begin{frame}
  \frametitle{Key Idea Behind the \MQ{} Modeling}
  \begin{itemize}
  \item Proposed by~\textcite{KS99}.
  \item Instead of guessing the kernel, we can attempt to explicitly construct
    it.
  \item If $\vec\upalpha$ is a solution,
    $\rank H_{\vec\upalpha} \le r \Longleftrightarrow \dim\left(\ker
      H_{\vec\upalpha}\right) \ge n - r$ $\Longleftrightarrow \exists\; n - r$
    linearly independent vectors in $\ker H_{\vec\upalpha}$.
  \item Write these vectors systematically as
    $\vec x^{(i)} = \begin{bmatrix} \vec e_i & \textcolor{RubineRed}{x_1^{(i)}}
      & \textcolor{RubineRed}{x_2^{(i)}} & \cdots &
      \textcolor{RubineRed}{x_r^{(i)}}\end{bmatrix}^T, \; 1 \le i \le n - r$,
    where $\vec e_i \in \mathbb{F}_q^{n - r}$ and the
    $\textcolor{RubineRed}{x_j^{(i)}}$'s are newly-introduced variables.
\end{itemize}
\end{frame}
\begin{frame}
  \frametitle{The \MQ{} System}
  Therefore, we can model a \MR{} instance as an \MQ{} system:
  \begin{equation}\label{eq:mq-sys}
    \left(\sum\limits_{i = 1}^{m}\textcolor{Blue}{\beta_i} \matr{M}_i - \matr{M}_0\right)
    \begin{bmatrix}
      1 & 0 & \cdots & 0 \\
      0 & 1 & \cdots & 0 \\
      \vdots & \vdots & \ddots & \vdots \\
      0 & 0 & \cdots & 1 \\
      \textcolor{RubineRed}{x^{(1)}_1} & \textcolor{RubineRed}{x^{(2)}_1} &
      \cdots & \textcolor{RubineRed}{x^{(n - r)}_1} \\
      \vdots & \vdots & \ddots & \vdots \\
      \textcolor{RubineRed}{x^{(1)}_r} & \textcolor{RubineRed}{x^{(2)}_r} &
      \cdots & \textcolor{RubineRed}{x^{(n - r)}_r}
    \end{bmatrix} = \matr{0}
  \end{equation}
  \eqref{eq:mq-sys} is a quadratic system of $\eta (n - r)$ equations in
  $r (n - r) + m$ variables.
\end{frame}
\begin{frame}
  \frametitle{Solving the \MQ{} System}
  \begin{itemize}
  \item The best method we have for solving multivariate polynomial systems of
    equations are lex Gröbner bases.
  \item Gröbner bases are defined w.r.t.\ monomial orderings.  A lex Gröbner
    basis can be thought of as a generalization of Gaussian elimination.
  \item The theoretical complexity of computing a Gröbner basis for a system
    with $m$ equations in $n$ variables is
    $\mathcal{O}\left(m \binom{n +
        d_{\mathrm{reg}}}{d_{\mathrm{reg}}}^\omega\right)$, where
    $d_{\mathrm{reg}}$ is the maximum degree reached during the computation and
    $2 \le \omega \le 3$ is the exponent in the complexity of matrix
    multiplication.
  \item The system given in~\eqref{eq:mq-sys} exhibits certain structural
    properties (it is formed by bilinear equations), so the complexity observed
    in practice is much lower.
  \end{itemize}
\end{frame}

\subsection{Implementation Details}
\begin{frame}
  \frametitle{Implementation of the Attacks}
  \begin{itemize}
  \item The implementations are done in SageMath and follow the theoretical
    foundations in a straightforward manner.
  \item The kernel attack is a simple implementation of~\cref{alg:ka}.
  \item Gröbner basis computation is done using the \textsc{Singular} procedure
    \href{https://www.singular.uni-kl.de/Manual/4-0-2/sing_403.htm}{\texttt{stdfglm}}.
    Internally, it uses the $\mathrm{F}_4$ algorithm to compute a Gröbner basis
    w.r.t.\ a degrevlex ordering, and then converts it to a lex ordering using
    the $\mathrm{FGLM}$ algorithm.  Once the Gröbner basis is computed,
    solving~\eqref{eq:mq-sys} is trivial and handled by SageMath's
    \texttt{variety()} method, which computes the affine variety of an ideal.
  \end{itemize}
\end{frame}

\section{Zero-Knowledge Authentication Based on MinRank}
\subsection{The Protocol}
\begin{frame}
  \frametitle{Key Idea Behind the Protocol}
  The protocol was proposed by~\textcite{Cou01}.  The key idea is stated in the
  following lemma.
  \begin{lemma}\label{lem:mask}
    Let $\matr{M}$ be an $\eta \times n$ matrix of rank $r \le \min(\eta, n)$.
    Let $\matr{S}$ and $\matr{T}$ be two uniformly distributed random
    nonsingular matrices of orders $\eta$ and $n$ resp.  Then $\matr{S M T}$ is
    uniformly distributed among all $\eta \times n$ matrices of rank $r$.
  \end{lemma}
  The takeaway is that a MinRank solution can be effectively \emph{masked} by
  two isomorphisms.  In order to force a prover to ``play by the rules,'' a
  collision-resistant hash function $\hash$ is used to make commitments.
\end{frame}
\begin{frame}
  \frametitle{The Prover Setup}
  \begin{enumerate}
  \item A uniformly chosen random combination $\vec\upbeta^{(1)}$ of the
    $\matr{M}_i$'s.
    $\matr{N}_1 = \sum\nolimits_{i=1}^{m} \beta_i^{(1)} \matr{M}_i$.
  \item Let $\vec\upbeta^{(2)} = \vec\upalpha + \vec\upbeta^{(1)}$, where
    $\vec\upalpha$ is the MinRank solution a legitimate prover should have
    access to.  $\matr{N}_2 = \sum\nolimits_{i=1}^{m} \beta_i^{(2)} \matr{M}_i$.
  \item Random nonsingular matrices $\matr{S}$ and $\matr{T}$, and a completely
    random matrix $\matr{X}$.
  \item The prover commits the hash values of the
    $(\matr{S}, \matr{T}, \matr{X})$ triple, and of
    $\matr{S}\matr{N}_1\matr{T} + \matr{X}$ and
    $\matr{S}\matr{N}_2\matr{T} + \matr{X} - \matr{S}\matr{M}_0\matr{T}$.
  \end{enumerate}
\end{frame}
\begin{frame}
  \frametitle{The Verifier}
  The verifier sends a random query ($\mathcal{Q} \sample \{0,1,2\}$) and
  either:
  \begin{itemize}
  \item Checks the committed hashes of the $(\matr{S}, \matr{T}, \matr{X})$
    triple and one of the $\matr{N}_i$'s; or
  \item Checks the committed hashes of $\matr{N}_1$, $\matr{N}_2$, \emph{and the
      rank of}
    $\matr{S}\matr{N}_2\matr{T} + \matr{X} - \matr{S}\matr{M}_0\matr{T} -
    \matr{S}\matr{N}_1\matr{T} + \matr{X} =
    \matr{S}\left(\sum\nolimits_{i=1}^{m} \alpha_i \matr{M}_i -
      \matr{M_0}\right) \matr{T}$.  This step is the backbone of the
    authentication, as by the previous lemma it remains a solution to the
    MinRank instance.
  \end{itemize}
  The protocol is \emph{black box zero-knowledge} with a cheating probability of
  $\frac 2 3$.  A prover authenticating herself means either
  \textcolor{RubineRed}{solving the \NP-complete problem of MinRank}, or
  \textcolor{RubineRed}{finding a collision in the hash function $\hash$} and
  playing ``dishonestly.''  Authentication is carried out in multiple rounds and
  is successful if and only if each round is successful.
\end{frame}
\subsection{Implementation Details}
\begin{frame}
  \frametitle{Implementation of the Protocol}
  \begin{itemize}
  \item The implementation follows the description of the protocol.  It is built
    around two objects, \texttt{Prover} and \texttt{Verifier} who are each
    associated to \texttt{MinRankInstance} objects.
  \item Legitimate provers are represented as \texttt{LegitimateProver} objects
    and can be given access to \texttt{MinRankInstance} objects.
  \item Instance generation is done according to the algorithm outlined
    in~\autocite{Cou01}, i.e.\ instances are generated such that both the
    $\matr{M}_i$'s and the solution $\vec\upalpha$ are uniformly distributed.
  \item There is no strict concept of public/private keys in the toy
    implementation, but in practice the keys are quite short as most of their
    parts can be generated by a pseudo-random generator from a shared seed.
  \end{itemize}
\end{frame}

\begin{frame}
  \frametitle{Performance}
  \begin{itemize}
  \item Instance generation is relatively fast: generating \num{10000} instances
    $m = 10, \eta = n = 6, r = 3, q = 65521$ required $\SI{10.252}{\second}$.
  \item Authentication performance depends largely on the parameter set
    (parameter sets A and C include few matrices over $\mathbb{F}_{65521}$,
    while D includes many matrices over $\mathbb{F}_2$).
  \end{itemize}
  \begin{table}[h]
    \centering
    \begin{tabular}{c S S}
      \toprule
      Parameter set~\autocite{Cou01} & {Time (legitimate)}
      & {Time (illegitimate)} \\
      \midrule
      A & 18.349 & 1.763 \\
      C & 133.610 & 11.450 \\
      D & 1050.127 & 91.196 \\
      \bottomrule
    \end{tabular}
  \end{table}
\end{frame}

\begin{frame}{References}
  \printbibliography
\end{frame}
\end{document}

%%% Local Variables:
%%% mode: latex
%%% TeX-master: t
%%% End:
