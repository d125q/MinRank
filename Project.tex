\documentclass{article}

%
% Packages
%

\usepackage[utf8]{inputenc}
\usepackage[T1]{fontenc}
\usepackage[english]{babel}
\usepackage[svgnames, psnames, x11names, table]{xcolor}
\usepackage{textcomp}
\usepackage[hmargin=.93in, vmargin=.93in, a4paper]{geometry}
\usepackage[babel, activate={true, nocompatibility}, final, tracking=true,
kerning=true, spacing=true]{microtype}
\usepackage{mathtools, amsthm, amssymb, xfrac, mathrsfs, cancel, upgreek, bm}
\usepackage{complexity}
\usepackage{graphicx}
\usepackage{caption,subcaption}
\usepackage[unicode, colorlinks, breaklinks]{hyperref}
\usepackage{smartref}
\usepackage[nameinlink]{cleveref}
\usepackage[operators, primitives]{cryptocode}
\usepackage{xr}
\usepackage{array}
\usepackage{enumitem}
\usepackage[binary-units, detect-all, detect-display-math]{siunitx}
\usepackage{tcolorbox}
\usepackage[newfloat]{minted}
\usepackage{algorithm, algpseudocode, varwidth}
\usepackage{booktabs}
\usepackage[bottom]{footmisc}
\usepackage{footnotebackref}
\usepackage[hyperref, backref, backend=biber, style=alphabetic, eprint=false]
{biblatex}
\usepackage{lmodern}

%
% Package options
%

\addbibresource{Bibliography.bib}
\renewcommand*{\bibfont}{\small}
\AtEveryBibitem{%
  \ifentrytype{manual}{%
  }{%
    \clearfield{url}%
    \clearfield{urldate}%
  }%
}

\SetTracking{encoding={*}, shape=sc}{40}
\graphicspath{{figure/}}

\DeclareMathOperator{\rank}{rank}
\DeclareMathOperator{\GF}{GF}

\newclass{\MR}{MR}

\renewcommand{\mkbibnamefirst}[1]{\textsc{#1}}
\renewcommand{\mkbibnamelast}[1]{\textsc{#1}}
\renewcommand{\mkbibnameprefix}[1]{\textsc{#1}}
\renewcommand{\mkbibnameaffix}[1]{\textsc{#1}}

\theoremstyle{remark}
\newtheorem*{remark}{Remark}
\theoremstyle{definition}
\newtheorem{definition}{Definition}
\theoremstyle{plain}
\newtheorem{theorem}{Theorem}
\newtheorem{lemma}{Lemma}
\newlength{\savedcolsep}

\renewcommand*{\vec}[1]{\bm{\mathrm{#1}}}
\newcommand{\matr}[1]{\bm{\mathrm{#1}}}
\newcommand{\MQ}{\ensuremath{\mathcal{MQ}}}
\newcommand{\STX}{\ensuremath{\overline{\matr{STX}}}}

\hypersetup{
  pdfinfo={
    Title={The MinRank Problem: Survey, Implementation, and One Application},
    Subject={Cryptography},
    Author={Dario Gjorgjevski},
    Keywords={Multivariate Cryptography, MQ Cryptosystems, MinRank, Gröbner
      Bases, Linear Algebra, Algebraic Cryptanalysis}
  },
  linkcolor=MediumBlue,
  urlcolor=DarkCyan,
  citecolor=ForestGreen
}

%
% Title
%

\title{The MinRank Problem: Survey, Implementation, and One Application}
\author{Dario Gjorgjevski\\
  \href{mailto:gjorgjevski.dario@students.finki.ukim.mk}
  {\texttt{gjorgjevski.dario@students.finki.ukim.mk}}}
\date{\textit{Project Work in Cryptography}\\[0.5\baselineskip]2015/2016}

%
% Document
%
 
\begin{document}
\maketitle

\section{Introduction}\label{sec:intro}
Multivariate cryptography is the generic term for asymmetric cryptographic
primitives based on multivariate polynomials over a finite field $\mathbb{F}_q$.
In the case when the degree of these polynomials is $2$, we refer to them as
multivariate quadratic (\MQ).  Note that the quadratic case is as complex as the
general case (up to a polynomial-time reduction).  The most interesting one-way
function in this context is the evaluation of multivariate polynomials.  Namely,
given
$\vec{f} = \left(f_1(x_1, \ldots, x_n), \ldots, f_\eta(x_1, \ldots, x_n)\right)
\in \mathbb{K}[x_1, \ldots, x_n]^\eta$ (where $\mathbb{K}$ is the ground field),
the one-way function is defined as follows:
\begin{equation}\label{eq:mq-func}
  F : \vec{x} = (x_1, \ldots, x_n) \in \mathbb{K}^n \mapsto \left(f_1(x_1,
    \ldots, x_n), \ldots, f_\eta(x_1, \ldots, x_n)\right)\text{.}
\end{equation}
The problem of inverting~\eqref{eq:mq-func} can be shown to be $\NP$-hard by
noticing that there is a trivial polynomial-time reduction from well-known
$\NP$-complete problems such as $3\text{-}\SAT$.

The MinRank problem ($\MR$) is a fundamental problem in linear algebra of
finding a low-rank linear combination of matrices.  It was first stated and had
its complexity analyzed in~\autocite{BFS99}.  The importance of the MinRank
problem in cryptography stems from the fact that it was shown to be very closely
related to several \MQ{} cryptosystems such as HFE~\autocite{Pat96,KS99} and
TTM~\autocite{Moh99, GC00}.  In fact, as we shall see in~\cref{sec:mq-attk},
$\MR$ instances can themselves be modeled as \MQ{} systems.

In this essay I will attempt to provide a survey of the MinRank problem and some
known methods for solving it, as well as look at the idea of using $\MR$
instances as the basis of a \emph{zero-knowledge authentication protocol}.

\section{The MinRank Problem}\label{sec:mr}
\begin{definition}[MinRank over a field]\label{def:mr}
  Let $\matr{M}_0;\, \matr{M}_1, \ldots, \matr{M}_m$ be matrices in
  $\mathcal{M}_{\eta \times n}(\mathbb{K})$.  The MinRank problem instance
  $\MR\left(m, \eta, n, r, \mathbb{K};\, \matr{M}_0;\, \matr{M}_1, \ldots,
    \matr{M}_m\right)$ asks us to find an $m$-tuple
  $\vec{\upalpha} = (\alpha_1, \ldots, \alpha_m) \in \mathbb{K}^m$ such that
$$
\rank\left(\sum\limits_{i = 1}^{m} \alpha_i \matr{M}_i - \matr{M}_0\right) \le
r\text{.}
$$
\end{definition}
\begin{remark}
  In practice, we will mostly be concerned with the case when $\mathbb{K}$ is a
  finite field, i.e., $\mathbb{K} = \mathbb{F}_q$.
\end{remark}
\begin{theorem}[Complexity of MinRank,
  ~\autocite{BFS99,Cou01}]\label{thm:mr-comp}
  The MinRank problem is $\NP$-complete.
  \begin{proof}[Proof sketch]
    There exists an effective algorithm to encode \emph{any} system of
    multivariate polynomial equations as a MinRank instance.  Another idea is to
    provide a reduction from the Syndrome Decoding problem to MinRank.
  \end{proof}
\end{theorem}

Note that the proof sketch of~\cref{thm:mr-comp} implies that the MinRank
problem is \emph{very general}, i.e., various other computationally hard
problems of practical importance can be reduced to solving $\MR$ instances.

\subsection{Solving MinRank Instances}
The two most practical and most widely documented methods of solving $\MR$
instances are:
\begin{itemize}[noitemsep]
\item The so-called ``kernel attack'';
\item Modeling $\MR$ instances as \MQ{} systems.
\end{itemize}

\subsubsection{The Kernel Attack}\label{sec:kern-attk}
This was the first non-trivial attack against MinRank and was proposed
by~\textcite{GC00}.  The main idea is that instead of guessing $\vec{\upalpha}$,
we can guess the kernel of the linear combination of the $\MR$ matrices (as
given in~\cref{def:mr}).  Then, assuming that the kernel was guessed correctly,
we can actually \emph{solve} for $\vec{\upalpha}$.

More formally, let
$\MR\left(m, \eta, n, r, \mathbb{F}_q;\, \matr{M}_0;\, \matr{M}_1, \ldots,
  \matr{M}_m\right)$ be a given $\MR$ instance.  For some parameter
$\vec{\upbeta} \in \mathbb{F}_q^m$, denote by $H_{\vec{\upbeta}}$ the linear
combination $\sum\nolimits_{i=1}^{m}\beta_i \matr{M}_i - \matr{M}_0$.  We would
like to have $\rank\left(H_{\vec{\upbeta}}\right) \le r$, i.e., $\vec{\upbeta}$
to be a solution.  If that were the case, then by the rank-nullity theorem of
linear algebra, we would have $\dim(\ker H_{\vec{\upbeta}}) \ge n - r$.

We can proceed by randomly choosing $k$ vectors
$\vec{x}^{(i)} \in \mathbb{F}_q^n,\ 1 \le i \le k$.  If these vectors are such
that they fall into the kernel of the $\MR$ instance, then solving for
$\vec{\upbeta}$ in
\begin{equation}\label{eq:kern-attk}
  \setlength{\arraycolsep}{0pt}
  \newcolumntype{B}{>{{}}c<{{}}}
  \left\{
    \begin{array}{l B l}
      H_{\vec{\upbeta}} \vec{x}^{(1)} & = & \vec{0} \\
      H_{\vec{\upbeta}} \vec{x}^{(2)} & = & \vec{0} \\
                                      & \vdots & \\
      H_{\vec{\upbeta}} \vec{x}^{(k)} & = & \vec{0}
    \end{array}
  \right.
\end{equation}
would allow us to retrieve the instance's solution.  \eqref{eq:kern-attk} is a
system in $k \eta$ equations and $m$ unknowns.  Since we
want~\eqref{eq:kern-attk} to never be underdetermined and have its numbers of
equations and unknowns match as closely as possible, we can choose
$k = \ceil{\frac m \eta}$.

It was established that there are at least $q^{n - r}$ vectors in
$\ker H_{\vec{\upbeta}}$ whenever $\vec{\upbeta}$ is a solution.  Having that in
mind, we can see that
\begin{equation}\label{eq:kern-prob}
  \Pr\left\{\left(x^{(1)},\ldots,x^{(k)}\right) \in \ker H_{\vec{\upbeta}}\right\}
  \ge q^{-k r} = q^{-\ceil{\frac m \eta} r}\text{.}
\end{equation}
What~\eqref{eq:kern-prob} tells us is that on average we have to guess \& solve
$q^{\ceil{\frac m \eta} r}$ times.  Accounting for the complexity of the
solving~\eqref{eq:kern-attk}, we get an overall complexity of
$\mathcal{O}\left(m \left(\ceil{\frac m \eta} \eta\right)^2 q^{\ceil{\frac m
      \eta} r}\right)$.  The takeaway is that this attack provides a significant
speedup over a brute-force search whenever $\ceil{\frac m \eta} r \ll m$.

\subsubsection{\MQ{}-Solving Attacks}\label{sec:mq-attk}
Modeling the MinRank problem as an \MQ{} system was first proposed
by~\textcite{KS99}.  The key idea here is that instead of guessing, we try to
explicitly construct the kernel of the linear combination of the $\MR$ matrices.

Again, let
$\MR\left(m, \eta, n, r, \mathbb{F}_q;\, \matr{M}_0;\, \matr{M}_1, \ldots,
  \matr{M}_m\right)$ be a given $\MR$ instance and $H_{\vec{\upbeta}}$ be
defined as in~\cref{sec:kern-attk}.  We already saw that whenever
$\vec{\upbeta}$ is a solution, there are at least $n - r$ linearly independent
vectors in $\ker H_{\vec{\upbeta}}$.  Proceed by trying to fix these vectors as
follows:
\begin{equation}\label{eq:fix-vecs}
  \setlength{\savedcolsep}{\arraycolsep}
  \setlength{\arraycolsep}{0pt}
  \newcolumntype{B}{>{{}}c<{{}}}
  \begin{array}{l B l}
    \vec{x}^{(1)} & =
    & \setlength{\arraycolsep}{\savedcolsep}
      \begin{bmatrix}1 & 0 & \cdots & 0 & x^{(1)}_1 & \cdots & x^{(1)}_r\end{bmatrix}^T \\
                  & \vdots & \\
    \vec{x}^{(n - r)} & =
    & \setlength{\arraycolsep}{\savedcolsep}
      \begin{bmatrix}0 & 0 & \cdots & 1 & x^{(n - r)}_1 & \cdots & x^{(n - r)}_r \end{bmatrix}^T\text{.}
  \end{array}
\end{equation}
\begin{remark}
  \Cref{eq:fix-vecs} is true up to a \emph{change of basis}, so it might happen
  that it does not help us arrive at a solution.
\end{remark}

Bearing this in mind, we can see that the \MQ{} system
\begin{equation}\label{eq:mq-sys}
  \left(\sum\limits_{i = 1}^{m}\beta_i \matr{M}_i - \matr{M}_0\right)_{\eta
    \times n}
  \begin{bmatrix}
    1 & 0 & \cdots & 0 \\
    0 & 1 & \cdots & 0 \\
    \vdots & \vdots & \ddots & \vdots \\
    0 & 0 & \cdots & 1 \\
    x^{(1)}_1 & x^{(2)}_1 & \cdots & x^{(n - r)}_1 \\
    \vdots & \vdots & \ddots & \vdots \\
    x^{(1)}_r & x^{(2)}_r & \cdots & x^{(n - r)}_r
  \end{bmatrix}_{n \times (n - r)} = \matr{0}_{\eta \times (n - r)}
\end{equation}
over
$\mathbb{F}_q\left[\beta_1,\ldots,\beta_m,x^{(1)}_1,\ldots,x^{(n -
    r)}_r\right]$, consisting of $\eta (n - r)$ equations in $r (n - r) + m$
variables, accurately models the solution to the $\MR$ instance.  The complexity
of solving~\eqref{eq:mq-sys} varies greatly depending on the method used --
nevertheless, it is exponential in the worst case.  The authors
of~\autocite{KS99} proposed using \emph{relinearization}.
In~\cref{app:solv-groeb,app:solv-xl} two methods will be discussed:
\hyperref[app:solv-groeb]{one} based on Gröbner bases and affine varieties, and
\hyperref[app:solv-xl]{the other} based on \emph{eXtended Linearization} (XL).

\subsection{Zero-Knowledge Authentication Based on MinRank}\label{sec:zka}
The use $\MR$ instances as an underlying $\NP$-complete problem in constructing
an efficient zero-knowledge authentication protocol was proposed
by~\textcite{Cou01}.

\subsubsection{Preliminaries}\label{sec:pre}
\begin{definition}[Zero-knowledge proof]\label{def:zkp}
  A \emph{zero-knowledge proof} is a method by which one party (the
  \emph{prover} $\prover$) can prove to another party (the \emph{verifier}
  $\verifier$) that a statement is true, without conveying \emph{any}
  information apart from the fact that the statement is indeed true.
\end{definition}
A zero-knowledge proof must satisfy three properties:
\begin{enumerate}
\item \label{itm:cmp} \textbf{Completeness.} A legitimate prover always gets
  accepted.
\item \label{itm:snd} \textbf{Soundness.} An illegitimate prover is rejected
  with some fixed probability.
\item \label{itm:zk} \textbf{Zero-knowledge.} No verifier can extract any
  information from a prover, even in multiple interactions.
\end{enumerate}

\subsubsection{The Authentication Scheme}\label{sec:sch}
The key idea behind the authentication scheme is stated in~\cref{lem:rnk-mask}.
\begin{lemma}\label{lem:rnk-mask}
  Let $\matr{M}$ be an $\eta \times n$ matrix of rank $r \le \min(\eta, n)$.
  Let $\matr{S}$ and $\matr{T}$ be two uniformly distributed random nonsingular
  matrices of orders $\eta$ and $n$ resp.  Then $\matr{S M T}$ is uniformly
  distributed among all $\eta \times n$ matrices of rank $r$.
  \begin{proof}[Proof sketch]
    $\matr{S}$ and $\matr{T}$ correspond to compositions of elementary row and
    column operations resp., which preserve the rank of the matrix.  $\matr{M}$
    and $\matr{S M T}$ are said to be \emph{matrix equivalent}.
  \end{proof}
\end{lemma}

\Cref{lem:rnk-mask} tells us that given a $\MR$ instance, we can \emph{mask} its
solution by multiplying it from the left and the right with nonsingular
matrices.  All we need now is a way to force the prover to ``play by the
rules.''  We define the \textbf{public key} to be a hard\footnote{Meaning both
  the matrices $\matr{M}_0;\, \matr{M}_1, \ldots, \matr{M}_m$ and the solution
  $\vec{\upalpha}$ are uniformly distributed.  See~\cref{app:mr-inst-gen} for
  the exact algorithm.} $\MR$ instance
$\MR\left(m, \eta, n, r, \mathbb{F}_q;\, \matr{M}_0;\, \matr{M}_1, \ldots,
  \matr{M}_m\right)$, and the \textbf{private key} to be the solution
$\vec{\upalpha}$ to the $\MR$ instance as per~\cref{def:mr}, with
$\matr{M} = \sum\nolimits_{i = 1}^{m} \alpha_i \matr{M}_i - \matr{M}_0$.  The
prover is going to convince the verifier of her knowledge of $\matr{M}$ (and
$\vec{\upalpha}$).  The authentication is performed for multiple rounds: a
prover $\prover$ succeeds in verifying herself if and only if she successfully
answers all queries posed by the verifier $\verifier$.

In order to get the prover $\prover$ to commit to playing by the rules, we use a
collision-resistant hash function $\hash$ as a random oracle to which $\prover$
makes commitments.  \Cref{fig:round} shows one round of MinRank authentication.

\begin{theorem}[\autocite{Cou01}]\label{thm:mr-zk}
  The MinRank authentication scheme is a \emph{zero-knowledge proof}.
  \begin{proof}\leavevmode
    \begin{enumerate}
    \item \textbf{Completeness.} It is clear that a legitimate prover, i.e., one
      with a correct $\vec{\upalpha}$ always gets verified.
    \item \textbf{Soundness.} An illegitimate prover being able to get verified
      implies that she either solved the $\MR$ instance or found a collision in
      $\hash$.  Both of these are computationally unfeasible -- MinRank is
      $\NP$-complete (\cref{thm:mr-comp}) and $\hash$ is assumed to be
      collision-resistant. Since the queries $\mathcal{Q} \in \{1, 2\}$ can be
      answered by any non-cheating prover, we see that the probability of an
      illegitimate prover not being rejected is
      $\left(\frac 2 3 + \ComplexityFont{negl}\right)^{\#\text{rounds}}$.
    \item \textbf{Zero-knowledge.}  This is a direct corollary of
      \cref{lem:rnk-mask}.
    \end{enumerate}
  \end{proof}
\end{theorem}

\begin{figure}[t!]
  \centering \fbox{%
    \procedure{}{%
      \textbf{Verifier}\ \verifier \< \< \textbf{Prover}\ \prover \\
      \< \< \matr{S} \sample \mathrm{GL}_{\eta}\left(\mathbb{F}_q\right) \\
      \< \< \matr{T} \sample \mathrm{GL}_{n}\left(\mathbb{F}_q\right) \\
      \< \< \matr{X} \sample \mathcal{M}_{\eta \times
        n}\left(\mathbb{F}_q\right) \\
      \< \< \vec{\upbeta}^{(1)} \sample \mathbb{F}_q^m \\
      \< \< \matr{N}_1 \gets \sum\nolimits_{i = 1}^{m} \beta^{(1)}_i \matr{M}_i \\
      \< \< \vec{\upbeta}^{(2)} \gets \vec{\upalpha} + \vec{\upbeta}^{(1)} \\
      \< \< \matr{N}_2 \gets \sum\nolimits_{i = 1}^{m} \beta^{(2)}_i \matr{M}_i \\
      \< \< \STX \gets (\matr{S}, \matr{T}, \matr{X}) \\
      \< \sendmessageleft*[5cm]{\hash\left(\STX\right), \hash\left(\matr{S}
          \matr{N}_1 \matr{T} + \matr{X}\right), \\
        \hash\left(\matr{S} \matr{N}_2 \matr{T} + \matr{X} - \matr{S} \matr{M}_0
          \matr{T}\right)} \< \\
      \mathcal{Q} \sample \{0, 1, 2\} \< \< \\
      \< \sendmessageright*[5cm]{\text{send query }\mathcal{Q}} \< \pclb
      \pcintertext[dotted]{if $\mathcal{Q} = 0$} \\
      \< \sendmessageleft*[5cm]{%
        \matr{S} \matr{N}_1 \matr{T} + \matr{X}, \matr{S} \matr{N}_2 \matr{T} +
        \matr{X} - \matr{S} \matr{M}_0 \matr{T}
      } \< \\
      \verify\left(\hash\left(\matr{S} \matr{N}_1 \matr{T} +
          \matr{X}\right)\right)
      \< \< \\
      \verify\left(\hash\left(\matr{S} \matr{N}_2 \matr{T} + \matr{X} - \matr{S}
          \matr{M}_0 \matr{T}\right)\right) \< \< \\
      \begin{aligned}
        \matr{S M T} \gets {} & (\matr{S} \matr{N}_2 \matr{T} + \matr{X} -
        \matr{S} \matr{M}_0 \matr{T}) \pclb & {} - (\matr{S} \matr{N}_1 \matr{T}
        + \matr{X})
      \end{aligned} \< \< \\
      \verify\left(\rank\left(\matr{S M T}\right) = r\right) \< \< \pclb
      \pcintertext[dotted]{else if $\mathcal{Q} \in \{1, 2\}$} \\
      \< \sendmessageleft*[5cm]{%
        \STX, \vec{\upbeta}_\mathcal{Q}
      } \< \\
      \text{check that }\matr{S}\text{ \& }\matr{T}\text{ are nonsingular} \< \< \\
      \verify\left(\hash\left(\STX\right)\right) \< \< \\
      \matr{S}\matr{N}_\mathcal{Q}\matr{T} \gets \sum\nolimits_{i = 1}^{m}
      \beta^{(\mathcal{Q})}_i\matr{S} \matr{M}_i \matr{T}
      \< \< \\
      \begin{aligned}
        &\verify\left(\hash\left(\matr{S} \matr{N}_1 \matr{T} + \matr{X}\right)
        \right) & \quad\text{if }\mathcal{Q} = 1 \pclb
        &\verify\left(\hash\left(\matr{S} \matr{N}_2 \matr{T} + \matr{X} -
            \matr{S} \matr{M}_0 \matr{T}\right)\right) & \quad\text{if
        }\mathcal{Q} = 2
      \end{aligned} \< \< }}
  \caption{One Round of Affine MinRank Authentication}\label{fig:round}
\end{figure}

\section{Conclusion}
The MinRank problem is of crucial importance to many cryptographic schemes and
ideas.  We have seen that it embodies many computationally hard problems,
including those used in various \MQ{} schemes and even cryptosystems based on
decoding linear codes such as the McEliece cryptosystem.

Schemes such as these and the one discussed in~\cref{sec:sch} are of interest to
the world of ``post-quantum cryptography.''  The publication of Shor's
algorithm~\autocite{Sho94} means that if a quantum computer were to be
constructed, ``traditional cryptography''---cryptography based on integer
factorization and discrete logarithms---would become immediately insecure.

Furthermore, a few attacks on the MinRank problem were reviewed.  These attacks
appear to be well-suited for attacking the MinRank authentication scheme.

\printbibliography

\appendix
\section{Technical Documentation}\label{app:impl}
The implementation was written in SageMath~\autocite{Sag15}.  It includes three
files:
\begin{itemize}[noitemsep]
\item \texttt{mr\_auth.py}: this file provides MinRank instance generation and a
  toy implementation of the zero-knowledge authentication scheme;
\item \texttt{mr\_attk.py}: this file implements the kernel and \MQ{}-solving
  attacks;
\item \texttt{mr\_util.py}: this file provides some utility functions;
\item \texttt{mr\_test.py}: this file runs some tests to showcase the
  implementation.
\end{itemize}
All code was tested on an Intel i5-4670 CPU @ \SI{3.40}{\GHz} running an
ArchLinux operating system.

\subsection{Instance Generation}\label{app:mr-inst-gen}
Instances are generated according to~\cref{alg:mr-inst-gen}.  Internally they
are represented as \texttt{MinRankInstance} objects.  It is easy to see that
this algorithm generates a correct instance.  It is also relatively fast:
generating \num{10000} instances with $m = 10, \eta = n = 6, r = 3, q = 65521$
required $\SI{10.252}{\second}$.

\begin{algorithm}[t]
  \renewcommand{\algorithmicrequire}{\textbf{Input:}}
  \renewcommand{\algorithmicensure}{\textbf{Output:}}
  \caption{MinRank Instance Generation~\autocite{Cou01}}\label{alg:mr-inst-gen}
  \begin{algorithmic}[1]
    \Require $m, \eta, n, r, q \in \mathbb{N}_+$ ($r \le \min(\eta, n)$)
    \Ensure An instance $\MR\left(m, \eta, n, r, \mathbb{F}_q;\,
      \matr{M}_0;\, \matr{M}_1, \ldots, \matr{M}_m\right)$ with solution
      $\vec{\upalpha}$ and
      $\matr{M} = \sum\nolimits_{i = 1}^{m} \alpha_i \matr{M}_i - \matr{M}_0$
    \State $\matr{M}_0;\, \matr{M}_1, \ldots, \matr{M}_{m - 1} \sample
      \mathcal{M}_{\eta \times n}\left(\mathbb{F}_q\right)$
    \State $\matr{M} \sample \mathcal{M}^{\mathrm{rank} = r}_{\eta \times n}
      \left(\mathbb{F}_q\right)$
    \State $\vec{\upalpha} \sample \mathbb{F}_q^m \text{ s.t.\ }
      \alpha_m \ne 0$
    \State $\matr{M}_m \gets \left(\matr{M} + \matr{M}_0
      - \sum\nolimits_{i = 1}^{m - 1}\alpha_i \matr{M}_i\right) \mathbin{/}
      \alpha_m$ \Comment{This ensures that $\matr{M}_m$ is ``correct''
      w.r.t.\ $\vec{\upalpha}$ and $\matr{M}$}
    \State \Return $\MR\left(m, \eta, n, r, \mathbb{F}_q;\, \matr{M}_0;\,
      \matr{M}_1, \ldots, \matr{M}_m\right)$, $\vec{\upalpha}$, $\matr{M}$
  \end{algorithmic}
\end{algorithm}

\subsection{Running the Kernel Attack}\label{app:run-kern-attk}
The implementation is very straightforward and basically
follows~\cref{sec:kern-attk}.  In order to run the attack, invoke the
\texttt{run\_kernel} method of a \texttt{MinRankAttack} object.  There is an
optional boolean parameter \texttt{constrained\_run} which indicates whether the
checking should stop after $q^{\ceil{\frac m \eta} r}$ attempts; otherwise it is
done until a valid solution is found.

\subsection{Solving the \MQ{} System Using Gröbner Bases}\label{app:solv-groeb}
The authors of~\autocite{FLP08} proposed a one-to-one correspondence between the
affine variety
$$
V(\mathcal{I}_{\mathrm{KS}}) = \{\vec{x} \in \mathbb{F}_q^{r (n - r) + m} :
f(\vec{x}) = 0 \; \forall f \in \mathcal{I}_{\mathrm{KS}}\}
$$
and the solution of~\eqref{eq:mq-sys}, where $\mathcal{I}_{\mathrm{KS}}$ is the
ideal generated by the equations of~\eqref{eq:mq-sys}.  This allows us to use
the functionality of SageMath/\textsc{Singular} to compute a solution.

The attack can be performed by invoking the \texttt{run\_groebner} method of a
\texttt{MinRankAttack} object.  Internally, it uses the \textsc{Singular}
procedure
\href{https://www.singular.uni-kl.de/Manual/4-0-2/sing_403.htm}{\texttt{stdfglm}},
which computes a Gröbner basis w.r.t.\ a degree-reverse lexicography ordering,
and then converts it to lexicographic ordering using the FGLM algorithm.
Lexicographic ordering is needed in order to ensure a triangular form of the
system.  One can think of this method as a generalization of the Gaussian
elimination to \MQ{} systems~\autocite[ch.\ 2, 3]{CLO15}.

For tiny instances ($m, \eta, n \approx 7$), I found the kernel attack to be
more effective, possibly because of its simplicity.  However,~\autocite{FLP08}
states that the Gröbner basis attack is feasible even for parameter sets
proposed in the MinRank authentication scheme.  In particular, the theoretical
bound on the complexity of computing a Gröbner basis of a polynomial system with
$m$ equations in $n$ variables is given by
$\mathcal{O}\left(m \binom{n +
    d_{\mathrm{reg}}}{d_{\mathrm{reg}}}^\omega\right)$ where $d_{\mathrm{reg}}$
is the degree of regularity (i.e., the highest degree reached during the
computation), and $2 \le \omega \le 3$ is the exponent in the complexity of
matrix multiplication.  Despite this, the observed complexity was much smaller
and can in fact be bounded by a \emph{polynomial} due to the structure
of~\eqref{eq:mq-sys} (it is formed by bilinear equations).

A problem which may arise is $\mathcal{I}_{\mathrm{KS}}$ ending up \emph{not
  being radical}.  To eliminate that possibility, we append the field equations
to~\eqref{eq:mq-sys}, which ensures that
$\sqrt{\mathcal{I}_{\mathrm{KS}}} = \mathcal{I}_{\mathrm{KS}}$ while not
changing the solution set.

\subsection{Solving the \MQ{} System Using XL}\label{app:solv-xl}
Relinearization, as it was proposed by~\autocite{KS99} was found to be
inefficient as it introduced way too many additional variables.  To alleviate
that issue, the authors of~\autocite{CKPS00} proposed a simplified and improved
version of relinearization called \emph{eXtended Linearization} (XL).

The idea is to raise the degree of the system sufficiently enough to be able to
apply Gaussian elimination as if it were a linear system.  The hope is that this
step yields a univariate polynomial which can be solved and substituted back in.

The XL algorithm takes a degree parameter $D \ge 2$, which indicates the maximal
degree to which the system will be raised.  It can be run using the
\texttt{run\_xl} method of \texttt{MinRankAttack}.  Unfortunately, I found the
XL algorithm to cope poorly with \MQ{} systems arising from MinRank instances.

\subsection{Implementation of the Authentication Scheme}
For the purposes of the MinRank authentication scheme, two main objects are
provided:
\begin{itemize}[noitemsep]
\item A prover, represented by the \texttt{Prover} class;
\item A verifier, represented by the \texttt{Verifier} class.
\end{itemize}

Both these objects are tied to \texttt{MinRankInstance} objects.  The
\texttt{Prover} and \texttt{Verifier} classes mimic their corresponding roles as
shown in~\cref{fig:round}.  A typical round of execution is shown
in~\cref{lst:round}.  In order to provide a wrapper around the procedure shown
in~\cref{lst:round}, the function \texttt{authentication\_driver} may be used,
taking a prover, verifier, and (optionally) the number of rounds as parameters.
\begin{listing}[h]
\begin{minted}[frame=single,linenos]{python}
# Assume P is the prover and V the verifier
P.prepare_round()  # Compute everything the prover needs for a round
P.send_hashes(V)   # Send the hashes, i.e., make a commitment
V.send_query(P)    # Send the query to the prover who will answer it

# Alternatively, execute this function
authentication_driver(P, V, 35)  # Perform 35 rounds
\end{minted}
\caption{Executing One Round of MinRank Authentication}\label{lst:round}
\end{listing}

As a cryptographic hash function a built-in implementation of SHA-256 is
used\footnote{Objects are hashed by hashing their internal representation.  This
  is likely far from the optimal way, but it does serve its purpose; at least in
  a toy implementation such as this one.}.  Due to the toy's implementation
``closed world'' nature, there is \emph{no} concept of public and private keys.
Instead, a \texttt{Prover} subclass \texttt{LegitimateProver} represents a
prover who can be given ``access to'' a MinRank instance.  This is achieved by
invoking the \texttt{give\_access} method of \texttt{MinRankInstance}.

Interestingly, even in a real world implementation, the key sizes are quite
small.  As part of~\cref{alg:mr-inst-gen},
$\matr{M}_0;\, \matr{M}_1, \ldots, \matr{M}_{m - 1}; \, \matr{M}$ need not be
transmitted as they can be generated from a common $\SI{160}{\bit}$
seed~\autocite{Cou01}.  This leaves only $\matr{M}_m$ to be transmitted, giving
a \SI[parse-numbers=false]{160 + \eta n \log_2 q}{\bit} public key.  The private
key is \SI[parse-numbers=false]{m \log_2 q}{\bit}s long (needed for
$\vec{\upalpha}$).

\Cref{tbl:time} shows some time measurements of authentications with different
parameter sets.  We see that parameter set D, which includes \num{81}
$19 \times 19$ matrices over $\mathbb{F}_2$ is a lot slower than parameter sets
A and C, both of which include \num{10} matrices (of dimension $6 \times 6$ and
$11 \times 11$ resp.) over $\mathbb{F}_{65521}$.  As expected, we also notice
that a successful authentication takes more time than an unsuccessful one.

\begin{table}[b]
  \centering
  \caption{Time Needed for Authentications
    (\num{100} Authentications, \num{35} Rounds Each)}\label{tbl:time}
  \begin{tabular}{c S S}
    \toprule
    Parameter set~\autocite{Cou01} & {Time (legitimate $\P$, in \si{\second})}
    & {Time (illegitimate $\P$, in \si{\second})} \\
    \midrule
    A & 18.349 & 1.763 \\
    C & 133.610 & 11.450 \\
    D & 1050.127 & 91.196 \\
    \bottomrule
  \end{tabular}
\end{table}
\end{document}

%%% Local Variables:
%%% TeX-command-extra-options: "-shell-escape"
%%% TeX-master: t
%%% End:
